%%% Fiktivní kapitola s ukázkami citací

\chapter{Odkazy na literaturu}

Odkazy na literaturu vytváříme nejlépe pomocí příkazů
\verb|\citet|, \verb|\citep| atp.
(viz {\LaTeX}ový balíček \textsf{natbib}) a~následného použití
Bib{\TeX}u. V~matematickém textu obvykle odkazujeme stylem \uv{Jméno
autora/autorů (rok vydání)}, resp. \uv{Jméno autora/autorů [číslo
odkazu]}. V~českém/slovenském textu je potřeba se navíc vypořádat
s~nutností skloňovat jméno autora, respektive přechylovat jméno
autorky. Je potřeba mít na paměti, že standardní příkazy
\verb|\citet|, \verb|\citep|
produkují referenci se jménem autora/autorů v~prvním pádě a~jména
autorek jsou nepřechýlena.

Pokud nepoužíváme bib\TeX{}, řídíme se normou ISO 690 a zvyklostmi
oboru.

Jména časopisů lze uvádět zkráceně, ale pouze v~kodifikované podobě.

\section{Několik ukázek}

Mezi nejvíce citované statistické články patří práce Kaplana a~Meiera a~Coxe
\citep{KaplanMeier58, Cox72}. \citet{Student08} napsal článek o~t-testu.

Prof. Anděl je autorem učebnice matematické statistiky
\citep[viz][]{Andel98}. Teorii odhadu se věnuje práce
\citet{LehmannCasella98}. V~případě odkazů na specifickou informaci
(definice, důkaz, \dots) uvedenou v~knize bývá užitečné uvést
specificky číslo kapitoly, číslo věty atp. obsahující požadovanou
informaci, např. viz \citet[Věta 4.22]{Andel07} nebo \citep[viz][Věta
4.22]{Andel07}.

Mnoho článků je výsledkem spolupráce celé řady osob. Při odkazování
v~textu na článek se třemi autory obvykle při prvním výskytu uvedeme
plný seznam: \citet*{DempsterLairdRubin77} představili koncept EM
algoritmu. Respektive: Koncept EM algoritmu byl představen v~práci
Dempstera, Lairdové a~Rubina \citep*{DempsterLairdRubin77}. Při každém
dalším výskytu již používáme zkrácenou verzi:
\citet{DempsterLairdRubin77} nabízejí též několik příkladů použití EM
algoritmu. Respektive: Několik příkladů použití EM algoritmu lze
nalézt též v~práci Dempstera a~kol. \citep{DempsterLairdRubin77}.

U~článku s~více než třemi autory odkazujeme vždy zkrácenou formou:
První výsledky projektu ACCEPT jsou uvedeny v~práci Genbergové a~kol.
\citep{Genberget08}. V~textu \emph{nenapíšeme}: První výsledky
projektu ACCEPT jsou uvedeny v~práci \citet*{Genberget08}.
