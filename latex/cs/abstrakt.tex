%%% Šablona pro jednoduchý soubor formátu PDF/A, jako treba samostatný abstrakt práce.

\documentclass[12pt]{report}

\usepackage[a4paper, hmargin=1in, vmargin=1in]{geometry}
\usepackage[a-2u]{pdfx}
\usepackage[czech]{babel}
\usepackage[utf8]{inputenc}
\usepackage[T1]{fontenc}
\usepackage{lmodern}
\usepackage{textcomp}

\begin{document}

%% Nezapomeňte upravit abstrakt.xmpdata.

Tato práce se zabývá vícestupňovými stochastickými programy a zkoumá závislost hodnoty účelové funkce na struktuře vybraného scénářového stromu. Scénářové stromy jsou tvořeny moment matching metodou, je formulován mean-CVaR model a dále na historických finančních datech je natrénován agent pomocí hlubokého zpětnovazebního učení za účelem volby co nejlepší možné struktury scénářového stromu. Pro tento účel jsme naimplementovali vlastní prostředí pro trénování zpětnovazebního agenta. Dále jsme navrhli přidání penalizace do odměny agenta za účelem penalizace stromů s moc složitou strukturou. Zpětnovazebního agenta jsme potom porovnali s agentem, který volí strukturu stromu náhodně a ukázali jsme, že zpětnovazební agent dosahuje lepších výsledků. Dále jsme analyzovali strukturu stromů zvolených zpětnovazebním agentem.

\end{document}
