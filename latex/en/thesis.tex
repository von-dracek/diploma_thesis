%%% The main file. It contains definitions of basic parameters and includes all other parts.

%% Settings for single-side (simplex) printing
% Margins: left 40mm, right 25mm, top and bottom 25mm
% (but beware, LaTeX adds 1in implicitly)
\documentclass[12pt,a4paper]{report}
\setlength\textwidth{145mm}
\setlength\textheight{247mm}
\setlength\oddsidemargin{15mm}
\setlength\evensidemargin{15mm}
\setlength\topmargin{0mm}
\setlength\headsep{0mm}
\setlength\headheight{0mm}
% \openright makes the following text appear on a right-hand page
\let\openright=\clearpage

%% Settings for two-sided (duplex) printing
% \documentclass[12pt,a4paper,twoside,openright]{report}
% \setlength\textwidth{145mm}
% \setlength\textheight{247mm}
% \setlength\oddsidemargin{14.2mm}
% \setlength\evensidemargin{0mm}
% \setlength\topmargin{0mm}
% \setlength\headsep{0mm}
% \setlength\headheight{0mm}
% \let\openright=\cleardoublepage

%% Generate PDF/A-2u
\usepackage[a-2u]{pdfx}

%% Character encoding: usually latin2, cp1250 or utf8:
\usepackage[utf8]{inputenc}

%% Prefer Latin Modern fonts
\usepackage{lmodern}

%% Further useful packages (included in most LaTeX distributions)
\usepackage{amsmath}        % extensions for typesetting of math
\usepackage{amsfonts}       % math fonts
\usepackage{amsthm}         % theorems, definitions, etc.
\usepackage{bbding}         % various symbols (squares, asterisks, scissors, ...)
\usepackage{bm}             % boldface symbols (\bm)
\usepackage{graphicx}       % embedding of pictures
\usepackage{fancyvrb}       % improved verbatim environment
\usepackage[numbers]{natbib}         % citation style AUTHOR (YEAR), or AUTHOR [NUMBER]
\usepackage[nottoc]{tocbibind} % makes sure that bibliography and the lists
			    % of figures/tables are included in the table
			    % of contents
\usepackage{dcolumn}        % improved alignment of table columns
\usepackage{booktabs}       % improved horizontal lines in tables
\usepackage{paralist}       % improved enumerate and itemize
\usepackage{xcolor}         % typesetting in color
\usepackage[textsize=tiny]{todonotes}
\usepackage{algorithm}
\usepackage{algpseudocode}
\usepackage{subcaption}
\usepackage[page,toc,titletoc,title]{appendix}


\reversemarginpar
%%% Basic information on the thesis

% Thesis title in English (exactly as in the formal assignment)
\def\ThesisTitle{Optimal Choice of Scenario Tree using Reinforcement Learning}

% Author of the thesis
\def\ThesisAuthor{Bc. Jakub Vondráček}

% Year when the thesis is submitted
\def\YearSubmitted{2023}

% Name of the department or institute, where the work was officially assigned
% (according to the Organizational Structure of MFF UK in English,
% or a full name of a department outside MFF)
\def\Department{Department of Probability and Mathematical Statistics}

% Is it a department (katedra), or an institute (ústav)?
\def\DeptType{Department}

% Thesis supervisor: name, surname and titles
\def\Supervisor{doc.~RNDr.~Ing.~Miloš~Kopa,~Ph.D.}

% Supervisor's department (again according to Organizational structure of MFF)
\def\SupervisorsDepartment{Department of Probability and Mathematical Statistics}

% Study programme and specialization
\def\StudyProgramme{Probability, mathematical statistics and econometrics}
\def\StudyBranch{MPSP}

% An optional dedication: you can thank whomever you wish (your supervisor,
% consultant, a person who lent the software, etc.)
\def\Dedication{%
I am infinitely grateful to my supervisor doc.~RNDr.~Ing.~Miloš~Kopa,~Ph.D. for his expertise, patience and all the time and effort spent supervising my thesis and for providing the necessary GAMS license. 

I am also deeply indebted to my consultant RNDr.~Karel~Kozmík for infinitely many great ideas, suggestions and always being ready to help whenever I needed guidance.

I dedicate this thesis to my grandmother and my mother, as without them, I would be nowhere near as far along in life as I am today. Thank you.
}

% Abstract (recommended length around 80-200 words; this is not a copy of your thesis assignment!)
\def\Abstract{%
This thesis deals with multistage stochastic programs and explores the dependence of the obtained objective value on the chosen structure of the scenario tree. In particular, the scenario trees are built using the moment matching method, a multistage mean-CVaR model is formulated and a reinforcement learning agent is trained on a set of historical financial data to choose the best scenario tree structure. For this purpose, we implemented a custom reinforcement learning environment. Further an inclusion of a penalty term in the reward obtained by the agent is proposed to avoid scenario trees that are too complex. The reinforcement learning agent is then evaluated against an agent that chooses the scenario tree structure at random and outperforms the random agent. Further the structure of scenario trees chosen by the reinforcement learning agent is analyzed.
}

% 3 to 5 keywords (recommended), each enclosed in curly braces
	\def\Keywords{%
{Stochastic optimization}, {Multistage problem}, {Reinforcement learning}
}

%% The hyperref package for clickable links in PDF and also for storing
%% metadata to PDF (including the table of contents).
%% Most settings are pre-set by the pdfx package.
\hypersetup{unicode}
\hypersetup{breaklinks=true}

% Definitions of macros (see description inside)
\include{macros}

\usepackage{sectsty}
\allsectionsfont{\raggedright}

\renewcommand*{\cite}{\citet}

% Title page and various mandatory informational pages
\begin{document}
\include{title}

%%% A page with automatically generated table of contents of the master thesis

\tableofcontents

%%% Each chapter is kept in a separate file
\chapter*{Introduction}
\addcontentsline{toc}{chapter}{Introduction}
Stochastic programming is a branch of mathematical optimization that allows to account for uncertain parameters when solving mathematical programs, which led to the widespread adoption of stochastic programming in fields such as finance, transportation, scheduling and telecommunications, see \cite{stochasticprogrammingbible2009}. 

This makes it a very powerful tool, which however comes at a significant computational cost. Due to the fact that the random parameters may follow a continuous distribution, approximating such distributions by a discrete set of scenarios is necessary to even be able to formulate the model and also to be able to solve it in finite time. Even more demanding are so called multistage programs, which allow multiple decision periods. To be able to solve multistage programs, the scenarios approximating the continuous distributions in every stage are arranged in a scenario tree. The structure of this tree is very important for the obtained solution, as a tree that is very simple may not approximate the underlying distribution correctly, while a tree that is too complex suffers from extensive computational costs. 

This is the main idea of this thesis -- to discover whether it is possible to predict a scenario tree structure that is optimal with regard to the objective function and also potentially with regard to the complexity of the scenario tree. To solve this problem, we propose an experiment to train a reinforcement learning agent (we follow mainly \cite{sutton2018reinforcement}) using the solutions of a mean-CVaR model (defined for example in \cite{cvar_robust_mean_cvar_portfolio_optimization}) calculated using scenario trees generated from historical financial data. 

Chapters \ref{chap1}, \ref{chap2} and \ref{chapter3} provide the necessary theory for Multistage stochastic programming, the mean-CVaR model and Reinforcement learning respectively. The mean CVaR model formulated in Section \ref{section:endofhorizoncvar_scenario_formulation}, while certainly not novel, is of our own design.  The main contribution of this thesis is Chapter \ref{chapter4}, the pinnacle of this thesis, where we implement the experiment described above and analyse the results. To the best of our knowledge, such an experiment has not been proposed in the literature. Also a notable contribution is the compilation and standardization of notation for several machine learning algorithms in Chapter \ref{chapter3} from multiple different sources.
\chapter{Stochastic programming}


%
%Structure of thesis:
%Introduction
%Multistage stochastic programming problems
%-multistage idea, nonanticipativity constraints, deterministic equivalent
%-curse of dimensionality
%-scenario trees 
%    -moment matching, geometric brownian motion + clustering
%Risk measures
%-VaR, CVaR, minimisation, L shaped method, reformulating as a multistage program
%Reinforcement learning
%-introduction, comparison to other types (supervised etc), no loss function - inspiration from learning of animals and men
%-Basic idea
%-Multiarmed bandits
%Study
%-Formulation of the whole problem (idea of thesis)
%    -Idea how to evaluate
%    -Penalisation of result using complexity of tree with a hyperparameter - explain reasoning why we should penalise the result - why are complex scenario trees bad (long computation time, ...)?
%-Data
%-Computational problems, complexity, how long does solving each small problem take
%-Results of the study
%

%Stochastic programming 
%-Framework to model problems which involve uncertainty
%-Optimization problem where some or all parameters are uncertain in contrast to deterministic optimization
%-the goal is to find a solution which is feasible for all such data and optimal in some sense
%-Stochastic programming models are similar in style but take advantage of the fact that probability distributions governing the data are known or can be estimated
%-The goal is to find some policy that is feasible for all (or almost all) the possible parameter realizations
%and optimizes the expectation of some function of the decisions and the random variables
%-Areas where stochastic programming is used
%	-financial planning, airline scheduling, transportation (truck routes, daily milk delivery with random demand), management of power systems
	
In this chapter, we give an introduction to the theory of stochastic programming with particular focus on multistage programs. Most of the content in this chapter is based on the material covered in \cite[Chapter 1]{stochasticprogrammingbible} and \cite[Chapters 1-3]{stochasticprogrammingbible2009}.
\todo{Add more meat here}

\section{Stochastic programming}

\section{Multistage stochastic programming}
\subsection{Scenario trees}
\subsubsection{Methods for generation of scenario trees}
\chapter{Risk measures}
Risk is a notion that affects everyone and everyone intuituitively understands it, yet defining it in a precise way would be very hard, at least in the common sense. In the area of financial mathematics, the definition, or maybe should we say the quantification of risk has evolved over time. In the following, we will follow mainly \cite{leoppold_risk_measures} and \cite[p. 275-278]{cornuejols_tutuncu_2006} if not specified otherwise.

\section{Risk measures	}
Let $\mathcal{R}$ be a random variable representing returns of a portfolio. In the last century, one of the most popular risk measures was the variance of returns of an asset, used famously in the Nobel Prize winning model in \cite{markowitz}, where a porfolio selection model was formulated that maximised the expected return while minimising the variance in returns. To be precise, variance in returns is defined as
\begin{equation*}
\sigma^2=E(\mathcal{R} - \mathbb{E}\mathcal{R} )^2.
\end{equation*}
Although at the time the achievement of the Markowitz model was groundbreaking, the use of variance as a risk measure has been a subject of debate. The problem is that variance is symmetric and does not take into account the tails of the distribution of $\mathcal{R}$. To handle the symmetry problem, Markowitz further proposed the semivariance defined as
\begin{equation*}
\gamma=\mathbb{E}(max(\mathcal{R} - \mathbb{E}\mathcal{R},0))^2.
\end{equation*}
The problem is that the formula for semivariance is much more complex and does not simplify so easily. Another risk measure is the mean absolute deviation, defined as
\begin{equation*}
\mathrm{MAD}=\mathbb{E}\abs{\mathcal{R} - \mathbb{E}\mathcal{R}}.
\end{equation*}
Let us now introduce the notion of \textit{coherent measure} of risk, which was developed in \cite[Defintion 2.4.]{coherent_measures_of_risk} and aimed to provide a set of properties that a “nice” risk measure should satistfy.
\begin{defn}{Coherent measure of risk} \\
Let $\mathcal{V}$ be the set of real random variables. We say that a risk measure $\rho: \mathcal{V} \rightarrow \R$ is coherent if it satisfies the following properties:
\begin{enumerate}
	\item Monotony: $X, Y \in \mathcal{V}, X(\omega) \leq Y(\omega) \, \forall \omega \in \Omega \implies \rho(X) \geq \rho(Y)$.
	\item Subadditivity: $X, Y, X+Y \in \mathcal{V} \implies \rho(X+	Y) \leq \rho(X) + \rho(Y)$.
	\item Positive homogenity: $X \in \mathcal{V}, h \geq 0, hX \in \mathcal{V} \implies \rho(hX)=h\rho(X)$.
	\item Translation equivariance: $X \in \mathcal{V}, a \in \R \implies \rho(X+a)=\rho(X)-a$.
\end{enumerate}
\end{defn}
The properties have a quite nice interpretation. The monotony property implies that a portfolio $Y$ that is more favourable in all possible scenarios should have smaller risk compared to the less favourable portfolio $X$. The subadditivity property pertains to the notion of diversification, as it says that if we combine two portfolios $X$ and $Y$, the resulting combined portfolio should not be riskier than the portfolios separately. The positive homogenity property pertains to the notion of leverage -- if we leverage our portfolio by some factor $h\geq0$, the risk should change proportionally to $h$. Last is the property of translation equivariance, which says that increasing the value of portfolio by risk free $a$, the risk profile is decreased by $a$. \todo{is this translation equivariance property ok?}

Let us now introduce another risk measure, the \textit{value at risk}.
Using the notation developed in \cite{cornuejols_tutuncu_2006}, let $f(x,\Upsilon)$ be a loss function of a vector $x$ which may be considered as a portfolio and a random vector $\Upsilon$ which represents the unknown returns or other random aspects influencing the distribution of loss and define the random variable $\mathcal{L}(x,\Upsilon)=f(x,\Upsilon)$. We assume that the probability distribution of $\Upsilon$ is known and for simplicity that $\Upsilon$ has a probability density $p(y)$. The cumulative distribution function of $\mathcal{L}(x,\Upsilon)$ is then defined as
\begin{equation*}
\Psi(x,y)=\int_{f(x,\gamma) \leq y} p(\gamma) \, \mathrm{d}\gamma.
\end{equation*}

\begin{defn}{Value at risk \cite[p. 275]{cornuejols_tutuncu_2006}.}  \\
Let $\alpha$ be the chosen confidence level and let $\mathcal{L}(x,\Upsilon)$ have the meaning of loss distribution as defined above. Then the value at risk $VaR_{\alpha}(x)$ is defined as
\begin{equation*}
VaR_{\alpha}(x)=q_{\alpha}(x)
\end{equation*}
where $q_{\alpha}(x)=\mathrm{min}\{\gamma \in \R: \Psi(x,\gamma) \geq \alpha \}$ is the lower $\alpha$ quantile of distribution of $\mathcal{L}(x,\Upsilon)$. $\alpha$ is usually chosen as $0.95$ or $0.99$.
\end{defn}

\begin{figure}
  \includegraphics[width=\linewidth]{../img/VaR_CVaR_graph_theory.png}
  \caption{Illustration of $VaR_{0.95}$ and $CVaR_{0.95}$ (where $CVaR_{0.95}$ is labeled as ES). Image sourced from \cite[Figure 2.2.]{mcneil2015quantitative}.}
  \label{fig:VaR_CVaR_graph_theory}
\end{figure}

Unfortunately, value at risk also comes with several advantages and disadvantages. While it is simple to understand and globally accepted by regulators, it is not coherent in general (the subadditivity requirement is not fulfilled), it doesn't quantify the losses exceeding $VaR_{\alpha}(\mathcal{L})$ and it is not convex (it is hard to optimize a portfolio with regard to value at risk).

Due to the aforementioned disadvantages of value at risk, another risk measure was considered.
Considering the expected loss exceeding the $VaR_{\alpha}(\mathcal{L})$ level leads to the notion of \textit{Conditional value at risk}. \todo{Add conditional value at risk.}

\begin{defn}{Conditional value at risk \cite[p. 275]{cornuejols_tutuncu_2006}}
\label{cvar_definition}
Let $\alpha$ be the chosen confidence level and let $\mathcal{L}(x,\Upsilon)$ have the meaning of loss distribution as defined above and assume that $\mathbb{E}(\abs{\mathcal{L}(x,\Upsilon)})<\infty$. Then the conditional value at risk or expected shortfall $CVaR_{\alpha}(x)$ is defined as
\begin{equation}
CVaR_{\alpha}(x)=\frac{1}{1-\alpha}\int_{f(x,y) \geq VaR_{\alpha}(x)} f(x,y)p(y) \, \mathrm{d} y.
\end{equation}
\end{defn}
Compared to value at risk, conditional value at risk is a coherent risk measure (for the proof, see \cite[Example 2.26.]{mcneil2015quantitative}), but since value at risk is present in its definition, and thus optimizing a portfolio with regard to value at risk according to this definition suffers from many of the same problems as optimizing a portfolio with regard to value at risk. Therefore, a new method has been developed in \cite{Rockafellar2000OptimizationOC} that allows optimization of conditional value at risk without computing value at risk using linear programming.

\section{Minimising CVaR using scenarios}
In this section, we closely follow the exposition provided in \cite[p. 275-278]{cornuejols_tutuncu_2006}.
Let
\begin{equation}
\label{eq:cvar_approx}
F_{\alpha}(x,y)=y + \frac{1}{1-\alpha} \int_{f(x,\gamma) \geq y} (f(x,\gamma)-y)p(\gamma) \, \mathrm{d}\gamma.
\end{equation}
The function $F_{\alpha}(x,y)$ has three important properties:
\begin{lemma}{Properties of $F_{\alpha}(x,y)$ \cite[p. 276]{cornuejols_tutuncu_2006}.} \\
\label{lemma:properties_of_cvar_approx}
The following three properties hold for $F_{\alpha}(x,y)$:
\begin{enumerate}
	\item It is a convex function of $y$.
	\item $VaR_{\alpha}(x)=\underset{y}{\mathrm{argmin}} \, F_{\alpha}(x,y)$.
	\item $\underset{y}{\mathrm{min}} \, F_{\alpha}(x,y) = CVaR_{\alpha}(x)$.
\end{enumerate}
\end{lemma}
\begin{proof}
For the proof, see \cite[Theorems 1 and 2]{Rockafellar2000OptimizationOC}
\end{proof}
\subsection{CVaR formulation}
If we want to choose a portfolio $x$ that minimises $CVaR_{\alpha}(x)$, we can now do so by minimising $F_{\alpha}(x,y)$ over $x \in \mathcal{X}$ and $y$ (where $\mathcal{X}$ is some set of portfolios) thanks to the third property in Lemma \ref{lemma:properties_of_cvar_approx}. Of course, Equation \ref{eq:cvar_approx} is not particularly suitable for numerical computations. In practice, as was explained in detail in Chapter \ref{chap1}, the distribution of $\Upsilon$ is approximated using scenarios $\gamma_s, s=1,\dots,S$.

We can then calculate an approximation of $F_{\alpha}(x,y)$ as
\begin{equation}
\label{eq:cvar_approx_approx}
\hat{F}_{\alpha}(x,y)=y + \frac{1}{1-\alpha} \sum_{s=1}^S  \max (f(x,\gamma_s)-y,0).
\end{equation}
We have arrived at the optimization problem
\begin{equation}
\label{eq:cvar_optim_first}
\underset{x \in \mathcal{X}, y}{\min} \hat{F}_{\alpha}(x,y)= \underset{x \in \mathcal{X}, y}{\min} y + \frac{1}{(1-\alpha)S} \sum_{s=1}^S  \max (f(x,\gamma_s)-y,0).
\end{equation}
A trick can be used to turn Equation \ref{eq:cvar_optim_first} into a linear programming problem. If we create new variables $z_s \geq 0$ such that $z_s \geq f(x,\gamma_s)-y$, we can write:
\begin{alignat}{10}
& \underset{x \in \mathcal{X}, z_s \geq 0, y}{\min}  \, \, \, && y + \frac{1}{(1-\alpha)S} \sum_{s=1}^S z_s \\
&s.t. && \, z_s \geq f(x,\gamma_s)-y, s=1,\dots,S
\end{alignat}
which is a linear programming problem.
\subsection{Mean-CVaR formulation}
In this section, we present a more precise formulation for practical use adopted from \cite{cvar_robust_mean_cvar_portfolio_optimization}, particularly when we want to choose a portfolio that minimises CVaR and also allows for controlling the expected return or setting the degree of risk aversion.
\subsubsection*{Formulation with minimum expected return}
Let $x=(x_1,\dots,x_n)$ be a vector denoting of weights of each of $n$ assets in a portfolio and consider that $\mu=(\mu_1,\dots,\mu_n)$ is a random vector representing the returns of the assets. Consider $S$ scenarios, each with probability $p_s$ and let $r_s = (r_{1,s},\dots,r_{n,s})$ be the particular realisation of $\mu$ in scenario $s$ and let $r_0$ be the minimum required expected return. For simplicity, we do not allow short selling (condition $x_i \geq 0, i=1,\dots,n$). Then we can write
\begin{alignat}{10}
& \underset{x_i \geq 0 , z_s \geq 0, y}{\min}  \, \, \, && y + \frac{1}{(1-\alpha)} \sum_{s=1}^S p_s z_s, \label{cvar_expected_return}  \\
&s.t. && \, z_s \geq  -\sum_{i=1}^{n} x_i r_{i,s} -y, s=1,\dots,S, \nonumber \\
&  && \sum_{i=1}^{n} x_i \bar{R_i} \geq r_0, \nonumber \\
&  && \sum_{i=1}^{n} x_i = 1, \nonumber
\end{alignat}
where $\bar{R_i}=\sum_{s=1}^{S}p_s r_{i,s}$, which is still a linear programming problem.
\subsubsection*{Formulation using risk aversion}
Another equivalent formulation might be useful when the decision maker does not require a minimum expected return explicitly, but rather wants to set his risk aversion expectations. This can be achieved by introducing a risk aversion parameter $\lambda \geq 0$ and writing
\begin{alignat}{10}
& \underset{x_i \geq 0 , z_s \geq 0, y}{\min}  \, \, \, && \sum_{i=1}^{n} x_i \bar{R_i} + \lambda \left( y + \frac{1}{(1-\alpha)} \sum_{s=1}^S p_s z_s \right), \label{eq:cvar_risk_aversion} \\
&s.t. && \, z_s \geq  -\sum_{i=1}^{n} x_i r_{i,s} -y, s=1,\dots,S, \nonumber \\
&  && \sum_{i=1}^{n} x_i = 1. \nonumber
\end{alignat}
\section{Minimising CVaR using scenarios in multistage setting}
In the multistage setting, the problem is a bit more complicated. Since the returns now do not occur at one single time but rather it is a sequence of returns, the notion of a risk measure must be extended accordingly. For the purposes of this thesis, we focus on the \textit{end of horizon $CVaR$}, for more advanced topics such as \textit{Nested $CVaR$ model} or \textit{Sum of $CVaR$ model}, see the summary in \cite[Section 1.4.]{kozmikv_phdthesis}.

\subsection{End of horizon CVaR}

\begin{defn}{End of horizon $CVaR$.}

Consider Definition \ref{cvar_definition}. If we consider the loss distribution $\mathcal{L}(x,\Upsilon)$ to have the meaning of the distribution of cumulative loss over all stages, we call this the end of horizon $CVaR$.
\end{defn}
The definition of \textit{end of horizon $CVaR$} is the same as of regular $CVaR$, with the small difference that we now allow the decision maker to reallocate funds in each stage. 

\subsubsection{End of horizon CVaR - scenario formulation}
We now extend Formulations \ref{cvar_expected_return} and \ref{eq:cvar_risk_aversion} to the multistage case. Consider we want to optimise a portfolio consisting of $n$ stocks over $T$ stages and consider a scenario tree with $S$ leaves. The problem \ref{cvar_expected_return} can then be reformulated as Equation \ref{eq:cvar_multistage_expected_return}:

\begin{alignat}{10}
& \min \, \, \, && y + \frac{1}{(1-\alpha)} \sum_{s=1}^S p_s z_s, \label{eq:cvar_multistage_expected_return}  \\
&s.t. && \, z_s \geq  -tot_s -y, s=1,\dots,S, \nonumber \\
&  && \sum_{s=1}^{S} p_s tot_s \geq r_0, \nonumber \\
&  && \sum_{i=1}^{n} x_{i,t,s} = 1, \nonumber \\
& && w_{0,s}=1, s=1,\dots,S, \nonumber \\
& && w_{t,s}=\sum_{i=1}^{n} x_{i,t,s}, s=1,\dots,S, \label{eq:cvar_multistage_expected_return:wealth_distribution} \\
& && w_{t+1,s}=\sum_{i=1}^{n} r_{i,t,s} x_{i,t,s}, s=1,\dots,S, \label{eq:cvar_multistage_expected_return:wealth_increases} \\
& && tot_s = \frac{w_{s,T}}{w_{s,0}}, s=1,\dots,S, \nonumber \\
& && z_s \geq 0, s=1,\dots,S, \nonumber \\
& && x_{i,t,s} \geq 0, \forall s, \forall i, \forall t, \nonumber \\
& && y \in \R ,\nonumber \\
& && + \mathrm{nonanticipativity \, constraints}, \nonumber
\end{alignat}
where $w_{t,s}$ represents the wealth in scenario $s$ at time $t$ and $tot_s$ is the total return in scenario $s$. The initial wealth $w_{0,s}$ is set to 1 and at each stage, the wealth increases by the returns obtained in the current stage (Equation \ref{eq:cvar_multistage_expected_return:wealth_increases}) and is distributed again (Equation \ref{eq:cvar_multistage_expected_return:wealth_distribution}). $r_{i,t,s}$ is the return obtained from stock $i$ at stage $t$ in scenario $s$ (we consider returns indexed by $t$ to occur after the portfolio allocations $x_{i,t,s}$ are set, so that Equation \ref{eq:cvar_multistage_expected_return:wealth_increases} makes sense). 

\begin{rem}
We do not include the nonanticipativity constraints in the above formulation, since they must be specified explicitly according to the structure of the scenario tree. To illustrate the explicit formulation of nonanticipativity constraints, consider Figure \ref{fig:balanced_scenario_tree}. In the context of the above problem, there are 6 scenarios ($111, 112, 113, 121, 122, 123$). For this illustration, let $\mathbf{x}_{t,s}$ be the vector of allocations to each assets at time $t$ and in scenario $s$. The nonanticipativity constraints now read:
\begin{alignat}{10}
& && \mathbf{x}_{t1,111}=\mathbf{x}_{t1,112}=\mathbf{x}_{t1,113}=\mathbf{x}_{t1,121}=\mathbf{x}_{t1,122}=\mathbf{x}_{t1,123} \nonumber \\
& && \mathbf{x}_{t2,111}=\mathbf{x}_{t2,112}=\mathbf{x}_{t2,113},\mathbf{x}_{t2,121}=\mathbf{x}_{t2,122}=\mathbf{x}_{t2,123}. \nonumber
\end{alignat}
\end{rem}

Similarly, the problem \ref{eq:cvar_risk_aversion} can then be reformulated as Equation \ref{eq:cvar_multistage_risk_aversion}:

\begin{alignat}{10}
& \min  \, \, \, &&\sum_{s=1}^{S} p_s tot_s + \lambda \left( y + \frac{1}{(1-\alpha)} \sum_{s=1}^S p_s z_s \right), \label{eq:cvar_multistage_risk_aversion}  \\
&s.t. && \, z_s \geq  -tot_s -y, s=1,\dots,S, \nonumber \\
&  && \sum_{i=1}^{n} x_{i,t,s} = 1, \nonumber \\
& && w_{0,s}=1, s=1,\dots,S, \nonumber \\
& && w_{t,s}=\sum_{i=1}^{n} x_{i,t,s}, s=1,\dots,S, \label{eq:cvar_multistage_expected_return:wealth_distribution} \\
& && w_{t+1,s}=\sum_{i=1}^{n} r_{i,t,s} x_{i,t,s}, s=1,\dots,S, \label{eq:cvar_multistage_expected_return:wealth_increases} \\
& && tot_s = \frac{w_{s,T}}{w_{s,0}}, s=1,\dots,S, \nonumber \\
& && z_s \geq 0, s=1,\dots,S, \nonumber \\
& && x_{i,t,s} \geq 0, \forall s, \forall i, \forall t, \nonumber \\
& && y \in \R ,\nonumber \\
& && + \mathrm{nonanticipativity \, constraints}, \nonumber
\end{alignat}
where $\lambda \geq 0$ is a risk aversion parameter and the nonanticipativity constraints would again need to be provided explicitly according to the structure of the scenario tree.
\chapter{Reinforcement learning}
Reinforcement learning is a machine learning paradigm inspired by the natural learning process of humans -- learning by interacting with an environment. All actions we take in our daily lives are in some way punished or rewarded. For an example, consider touching a hot stove. An immediate negative reward (pain) is received and one learns quickly not to do it again. On the other hand, eating something sweet usually produces a feeling of pleasure (positive reward) and that makes us want to eat more sweets. Reinforcement learning methods work in pretty much the same way: an agent is placed in an artificial environment and based on the actions it takes, it receives rewards (positive or negative) and learns to perform the actions that yield the most positive rewards. This is in constrast to the other machine learning paradigms (supervised and unsupervised learning), where no environment exists and the model is learned by minimising some kind of loss over a given dataset. 

In the recent years, machine learning has seen a large surge in activity due to rising computational power and this has not avoided the field of reinforcement learning. Many large institutions and corporations have built teams that specialise in reinforcement learning and have produced groundbreaking results in many disciplines, ranging from beating the best player in the world in the game of Go (see \cite{alphago_paper}), solving the protein folding problem (see \cite{alphafold}), beating some of the best teams in Dota 2 (see \cite{openaifive}) or most recently, finding a faster matrix multiplication algorithm that current state of the art (see \cite{matrix_multiplication}).

In this chapter, we aim to provide the necessary exposition of reinforcement learning methods used in the computational part of this thesis. We mainly follow \cite{sutton2018reinforcement}.
\chapter{Optimal scenario tree selection}
\label{chapter4}
In this chapter we propose an experiment to find out if it is possible to predict the optimal scenario tree structure with regard to the objective function and also propose a way to control for the complexity of the scenario tree using reinforcement learning. For this purpose, we implemented the moment matching method for generation of scenario trees, the multistage mean-CVaR model and a reinforcement learning agent.
\section{Methods}
The whole implementation was programmed in Python (\cite[Version 3.11]{python}), mathematical optimization problems were implemented in GAMS (\cite[version 40.3.0]{GAMS}) using the Python API. Data were sourced from \cite{yahoo} using the \cite[version 0.1.74]{yfinance} package. The reinforcement agent was implemented using the \textit{Stable~baselines~3} package (\cite[version 1.6.2]{stable_baselines3}) and the environment was implemented using the \textit{gym} package (\cite[version 0.21.0]{openai_gym}).

\section{Data}
\label{section:data}
For our experiments, we used data obtained from \cite{yahoo} using the \cite[version 0.1.74]{yfinance} package. We downloaded historical weekly asset price data from 1.1.2000 to 31.12.2019 for 49 financial stocks given in Table \ref{table:stock_tickers_used}. We consider two indexing sets with regard to time, set \textit{train} (from 1.1.2000 to 31.12.2009) and set \textit{test} (from 1.1.2010 to 31.12.2019) and also consider two sets of assets, set $A$ (see Table \ref{table:stock_tickers_in_set_A}) and set $B$ (see Table \ref{table:stock_tickers_in_set_B}). This yielded us four distinct sets:
\begin{itemize}
\item (\textit{train}, $A$), denoted $TrA$,
\item (\textit{test}, $A$), denoted $TeA$,
\item (\textit{train}, $B$), denoted $TrB$,
\item (\textit{test}, $B$). denoted $TeB$,
\end{itemize}
and we write the set that contains all these sets as $\kappa=\{TrA, TeA, TrB, TeB\}$.
We trained the agent on the set $TrA$ and evaluate its performance on all four sets, to evaluate whether the agent is able to:
\begin{enumerate}
\item learn something from the training data (performance on $TrA$),
\item generalise to unseen assets in the same period (performance on $TrB$),
\item generalise to the training assets in the future (performance on $TeA$),
\item generalise to unseen assets in the future (performance on $TeB$).
\end{enumerate}

\begin{rem}
We are mainly interested in the performance of the agent on sets $TrA$ and potentially also $TrB$. If the agent was able to generalise across time, this would be a breaktrough finding. Such generalisation is however unlikely due to the possibility that the distribution of the data may be different in the \textit{test} time period.
\end{rem}

We set the investment horizon to 2.5 years and from each set in $\kappa$, we needed to obtain data for the moment matching method in such a way that the scenario trees were constructed with the same investment horizon independent of the number of stages. This was achieved by splitting the investment horizon into equisized parts based on the number of stages of the tree.
Particularly, for a scenario tree with a given number of stages (denoted as $T$), we split the investment period into $T$ equisized periods (we denote the length of these periods as $\mathcal{L}$), each corresponding to the given stage.

We then used the whole available 10 years of data to estimate the distribution of returns in a period of length  $\mathcal{L}$ by splitting the 10 years of data into equisized parts of length $\mathcal{L}$, calculating simple returns (this yielded $4T$ observations), and calculating the first four sample moments and correlations between each of the assets. These moments and correlations were then used as input for the moment matching method, where the obtained moments and correlations were used for generating each stage of the scenario tree.

\begin{rem}
We considered only stagewise independent scenario trees. This is in line with the data preparation we used above, as financial returns are generally considered independent when taken over a period of time that is longer than a few days. The shortest period we used is about $6$ months (which corresponds to an investment period of 2.5 years with 5 stages), which is much longer than a few days.
\end{rem}

\section{Environment}
For the purposes of training the reinforcement agent to choose the best scenario tree structure, we adapted the well known GridWorld environment to represent iterative stage by stage building of the scenario tree.
\begin{rem}
The GridWorld environment is a well known introductory environment for training reinforcement learning agents. It consists of a $n$ by $n$ grid, $n \in \N$, where the agent starts on a given tile (a position on the grid, i.e. for example the bottom left corner) and must reach a target tile and upon reaching the tile, it receives a reward. The agent can perform 4 actions -- move up, move down, move left and move right. None of the tiles apart from the target tile return rewards. An illustration of such an environment is given in Figure \ref{fig:gridworldenv_illustration}.
\end{rem}

\begin{figure}[H]
\centering
  \includegraphics[width=\linewidth / 3]{../img/gridworld_env_illustration.pdf}
  \caption{Illustration of a 4 by 4 GridWorld environment. The agent starts at the Initial tile and must reach the Target tile to receive a reward.}
  \label{fig:gridworldenv_illustration}
\end{figure}

We designed and implemented a custom GridWorld environment, which we call \textit{Tree Building Environment}.

\subsection{Tree Building Environment}
\label{section:treebuildingenv}
Tree Building Environment is an adaptation of the GridWorld environment to allow the agent to build scenario trees based on given predictors, which consist of return data of the assets that are used to build the scenario tree. 
The state is given as a 8 by 8 tree building grid (which starts filled with zeros only in the initial state) and the set of 9 predictors. 

At the beginning of each episode, between 7 and 10 assets are randomly chosen (with uniform probabilities for the number of assets) from the provided data (a set from $\kappa$) and the set of predictors is obtained at the beginning of each episode using the return data for the chosen assets for the whole time period and is constant throughout the episode. 

Each row in the tree building grid represents a part of the tree building process. The first row corresponds to the chosen number of stages in the tree. The following rows each represent the chosen branching (number of descendants of each node) in each stage, from the first to last.

In each state, the agent can take any action in the set $\{3,4,5,6,7\}$, which we from now on refer to as \textit{action set}. In the initial state, we allow the agent to perform only actions $\{3,4,5\}$ from the action set to choose the depth of tree and upon performing any of these actions, 1 is placed at the corresponding position in the first row. If the agent chooses action 6 or 7, it is forced to perform action 5\footnote{We do not allow the number of stages to be 6 or 7, as the tree is then considerably more complex and building it based on historical data and solving the resulting mean-CVaR model takes too much time for it to be practical for our experiments. The action 5 is forced for the reinforcement algorithm to be able to associate action 6 and 7 with the move to a state where action 5 was chosen.}.


\begin{figure}[H]
  \includegraphics[width=\linewidth]{../img/Treebuildingenv_graph.pdf}
  \caption{Tree Building Environment. State illustration when a 4 stage tree is generated with 5 children in the first stage, 6 in the second stage, 5 in the third stage and 4 in the fourth stage. Crosshatching represents invalid actions. Predictors are represented as an empty array, in reality they are populated with numerical data on the whole period returns, see Section \ref{subsection:predictors}.}
  \label{fig:treebuildingenv}
\end{figure}


In the following states, where the agent chooses the branching (the number of descendants) in each stage, the agent can perform any action in the action set and again upon performing each action, 1 is placed at the position of the chosen action in the next row (if the action is valid, see below). 

To obtain reasonable trees, we had to constrain the size of the tree the agent can take (the maximum number of scenarios). We always require that the tree must have at least 100 scenarios and at most 1200. In each state, we check if it is possible to perform the chosen action such that a final tree that remains within these limits is possible. If it is not possible, the action is considered invalid and the agent is forced to take the maximum\footnote{we use the maximum valid action, rather than a random action, so that the agent is able to associate that an invalid action results in taking a large branching in the given stage} valid action (where valid action refers to any action for which building a tree that stays in the given limits is possible). In case the chosen action was invalid, 1 is placed at the position of the forced maximum valid action.

When the agent has taken as many actions as the chosen depth of tree, the episode ends, the mean-CVaR problem is solved using the given scenario tree structure and the obtained reward is returned (see Section \ref{section:rewards}). An illustration of the environment can be found in Figure \ref{fig:treebuildingenv}.

\subsection{Predictors}
\label{subsection:predictors}
At the beginning of each episode, between 7 and 10 assets are randomly chosen (with uniform probabilities for the number of assets) from the provided data and simple returns are calculated for the whole time range. $\alpha$ is randomly chosen with uniform probabilities from the set $\{0.8, 0.85, 0.9, 0.95\}$ and the following predictors are provided to the agent:
\begin{enumerate}
\item $\alpha$,
\item Number of sampled assets,
\item Sample maximum return,
\item Sample minimum return,
\item Sample 0.75 quantile of returns,
\item Sample 0.5 quantile of returns,
\item Sample 0.25 quantile of returns,
\item Sample mean of returns,
\item Sample variance of returns.
\end{enumerate}
The predictors are then constant throught the entire episode.

\section{Rewards}
\label{section:rewards}
A reward is returned after every action the agents takes, which is 0 for every action in states that are not terminal and at the end of the episode, in the terminal state, a reward is returned based on the solved mean-CVaR problem using the scenario tree structure that is specified by the terminal state. 
\begin{rem}
Note that the fact that the reward is sparse (reward 0 is returned when not in a terminal state) is actually quite a common occurence in reinforcement learning. Reinforcement learning is, after all, designed exactly to handle such problems.
\end{rem}

We need to specify the reward in such a way that a higher reward corresponds to a more favourable value of the objective function. We need to consider that the mean-CVaR model, as formulated in Equation \ref{eq:cvar_multistage_risk_aversion}, is formulated using the distribution of loss. This means that obtaining a smaller objective value from the mean-CVaR model is beneficial. Denoting the obtained objective value from solving the problem in Equation \ref{eq:cvar_multistage_risk_aversion} as $\varsigma$, we have to write the reward given to the agent as
\begin{equation*}
r_{terminal} = -\varsigma(s_{terminal}),
\end{equation*}
where we add the subscript $terminal$ to emphasize that this reward is calculated at the terminal state of the episode and that the reward depends on the terminal state (which represents the scenario tree structure). The objective of the agent is then to maximise $r_{terminal}$.

\subsection{Penalty}
\label{subsection:penalty_subsection}
As was mentioned in Section \ref{section:curse_od_dimensionality}, choosing a scenario tree that has too many stages and too many descendants in each stage leads to a computationally intractable problem. On the other hand, choosing a scenario tree that is too small in terms of number of stages and with too few descendants in each stage may lead to a very rough approximation of the underlying continuous distributions, leading to results with high variability. 

To ameliorate these problems, we propose to include a penalty term in the reward $r_{terminal}$ which penalizes such scenario trees. Particularly, we propose that the penalty be dependent on the number of scenarios in the tree in the last stage, i.e. the number of leaves in the tree. It is not straightforward to represent the complexity of a tree due to the multidimensional structure, but we consider that using the number of leaves provides a good enough proxy for the complexity of the scenario tree, while being simple to implement. 

Denoting the number of leaves in the tree as $\Psi$ and the penalization function as $\delta(\Psi)$, we thus propose that the reward $r_{terminal}$ be penalized as follows
\begin{equation*}
r_{terminal} = -\varsigma(s_{terminal}) - \delta(\Psi(s_{terminal})),
\end{equation*}
where the dependence of $\Psi$ on $s_{terminal}$ stresses the fact that $\Psi(s_{terminal})$ is calculated from the scenario tree structure represented in $s_{terminal}$.

To penalize the scenario tree that is too complex, we propose a linear penalization $\delta_1$
\begin{equation*}
\delta_1(\Psi) = c \frac{\Psi - \Psi_{min}}{\Psi_{max} - \Psi_{min}},
\end{equation*}
where $c$ is a chosen coefficient (which must be chosen based on the magnitude of values obtained as solutions from solving the mean-CVaR problem) and $\Psi_{max}$ and $\Psi_{min}$ are the maximum and minimum allowed number of leaves in the scenario tree respectively.

%Penalisation $\delta_1$ penalizes the complexity, but does not take into account that a small number of leaves is also detrimental. We propose another penalisation function, denoted $\delta_2$, to deal with this problem:
%\begin{equation*}
%\delta_2(\Psi) = \frac{c}{\left[(\Psi_{max}-\Psi_{min})/2\right]^2} (\Psi - \frac{\Psi_{max}+\Psi_{min}}{2})^2,
%\end{equation*}
%where again $c$ is a parameter, $\Psi_{max}$ is the maximum allowed number of leaves in the scenario tree and $\Psi_{min}$ is the minimum allowed number of leaves in the scenario tree.

%The proposed $\delta_2$ penalisation is quadratic and centered at $(\Psi_{max}+\Psi_{min})/2$ where it is 0, $c$ is the maximum penalty achieved at points $\Psi_{max}$ and $\Psi_{min}$. $c$ again is a coefficient that must be chosen based on the desired magnitude of the penalty.

Of course, this is just one possible penalization function out of infinite possibilities. The shape of the penalty function can be adjusted based on the problem at hand (and the parameter $c$ has to be adjusted as well).

In this section, we proposed only a penalization function that penalizes trees that are too complex. It might make sense to penalise also trees that are too simple, but we do not use such a penalisation in this thesis, as we already have a lower bound set on the number of leaves in the tree in the environment (the lower bound is 100 scenarios).

\section{Implementation}
\subsection{Moment matching}
\label{subsection:moment_matching}
%Gülpinar N, Rustem G, Settergren R, Simulation and Optimization Ap-
%%proaches to Scenario Generatin, Journal of Economic Dynamics & Control,
%vol. 28, Elsevier Science, 2004 - Moment matching - sequential vs whole tree
We used the moment matching method in the form given in Definition \ref{defn:moment_matching_method} sequentially on each stage using the first four sample moments and correlations which were estimated as explained in Section \ref{section:data} with one small adjustment. 

When looking at the generated scenarios for larger numbers of descendants, we noticed that usually, only 3 scenarios with positive probabilities were generated and the rest had almost zero probability. This would fundamentally change the properties of scenario trees that we want to explore (dependence of objective function on tree size), since then we might think we are using a large tree, which in reality is much smaller due to the scenarios with zero probability. 

To counteract this effect, we added the constraint $p_j \geq 0.03$ to the implementation of Definition \ref{defn:moment_matching_method}, which solved the problem. With the notation developed in Definition \ref{defn:moment_matching_method}, the model now reads
\begin{alignat}{10}
& && && \underset{\substack{p_j, x_{i,j}, \\ j \in \{1,...,N\}, i \in I}}{\min} \sum_{i\in I} \sum_{k\in \mathcal{M}} \left(m_{i,k} - M_{i,k}\right)^2 + \sum_{(i, i') \in I, i < i'}(c_{i,i'}-C_{i,i'})^2 \nonumber \\
& s.t. && \sum_{j=1}^N p_j&&=1 \nonumber \\
& && m_{i,1}&&=\sum_{j=1}^N p_jx_{i,j}, i \in I \nonumber \\
& && m_{i,k}&&=\sum_{j=1}^N p_j(x_{i,j}-m_{i,1})^k, i \in I, k>1 \nonumber \\
& && c_{i,i'}&&=\sum_{j=1}^N(x_{i,j}-m_{i,1})(x_{i',j}-m_{i',1})p_j, i,i' \in I, i<i' \nonumber
\end{alignat}
\vspace{-0.5cm}
\begin{alignat}{10}
& 0  \leq x_{i,j}, i \in I, j=1,\dots,N, \nonumber \\
& \frac{3}{100} \leq p_j \leq 1, j=1,\dots,N. \nonumber
\end{alignat}

This means that we are generating stagewise independent balanced scenario trees, where the probabilities of each child node may vary, but are at least 0.03. This may lead to the fact that we are not able to account for scenarios with very small probability, which is a limitation, as financial data distributions are generally heavy tailed. However, for the purposes of this thesis, this assumption is not too restrictive.


\subsection{Mean-CVaR model}
The mean-CVaR model was implemented exactly as given in Equation \ref{eq:cvar_multistage_risk_aversion} using the scenario tree generated from the moment matching method, where we used the risk aversion parameter $\lambda=0.3.$
\subsection{Reinforcement agent}
We chose to use a tried and tested implementation of state of the art algorithms in the \textit{Stable Baselines 3} library (\cite{stable_baselines3}). We experimented with multiple architectures and algorithms implemented therein, particularly A2C and PPO, while eventually settling on using PPO in the results given in Section \ref{section:experimental_results}. Here we share our experience with training the reinforcement agent.

We first experimented with the algorithms using a toy environment (Tree Building Environment with synthetic predictors and rewards) to obtain some semblance of how long it takes to obtain a reward better than random guessing. We experimented with several neural net architectures and found out that PPO usually outperformed A2C (converged much faster) with the same neural net architecture. 

We also experimented with neural net architectures and found that even for very simple tasks (such as learning a different action based on the value of a single predictor $p$), a very nontrivial number of neurons in the hidden layers is required for the model to be able to solve the environment. Particularly, for a deterministic toy example where $p$ was randomly sampled from the set $\{0.1, 0.2\}$ and based on the given $p$ the best number of stages to take was $3$ if $p=0.1$ and $5$ if $p=0.2$, the reinforcement agent didn't learn anything within hudreds of thousands of timesteps, unless we used an architecture with at least two hidden layers of 128 and 64 neurons respectively, where the second layer is separate for the actor and the critic (which estimate the policy and the action value). Furthermore, we used ReLu activations between each layer. 

Due to the results obtained from the synthetic toy environment, we decided to use a very similar architecture as given above, where the only change is that we use 256 and 128 neurons in the hidden layers instead of 128 and 64, as the task we are trying to solve is much more difficult and stochastic. Unfortunately, due to the computational difficulties presented in Section \ref{section:computational_difficulties}, we couldn't experiment with multiple neural network architectures by doing hyperparameter optimization on the number of neurons in each layer. The final neural net architecture that we use is visualised in Figure \ref{fig:neuralnet}. 


With regard to the hyperparameters, we used the learning rate $\varphi=0.001$, discount factor $\gamma=1$ (which is used for the calculation of $\widehat{A}(a,s)$, see \cite[Section 5]{proximal_policy_optimization}) so that the agent is not penalised for using trees with more stages and performed $192$ timesteps per update of the parameters of the neural network. Otherwise, we used the default hyperparameters, particularly $c_1=0.5$, where $c_1$ is the value function coefficient given in Equation \ref{eq:ppo_performance_measure}).

\begin{figure}[H]
  \centering
  \includegraphics[width=\linewidth / 2]{../img/nn_diagram.pdf}
  \caption{Illustration of the neural network architecture used for training the PPO agent.}
  \label{fig:neuralnet}
\end{figure}

\section{Experimental results}
\label{section:experimental_results}
\input{experimental_results}

\newpage
\section{Computational difficulties}
\label{section:computational_difficulties}
The writing of this thesis has been plagued by computational difficulties from the start and we wish to share our experience with implementing all parts of this thesis here. 

First of all, the moment matching method, which seems quite easy to implement, required a significant amount of experimentation and trying different optimization frameworks and solvers to actually obtain a working implementation. We have tried several python packages with open source solvers (most notably \textit{SciPy} (\cite{scipy}) , \textit{Mystic} (\cite{mystic}) and \textit{Gekko} (\cite{gekko}) and the IPOPT solver) and none of them produced a suitable result despite correct implementation due to low strength of the open source solvers. We finally settled on using GAMS with the CONOPT solver which worked out quite nicely. This however required us to connect our Python code to GAMS using the GAMS Python API, which meant that we could not use a compute cluster due to licensing limitations. 

Since we already had the dependence on GAMS, we also implemented the mean-CVaR model in GAMS using the CPLEX solver. While the implementation itself in GAMS was not terribly difficult, bending all data in the correct way and formulating the nonanticipativity constraints correctly took significant effort.

Lastly, the reinforcement learning part. This part was plagued by slow training and therefore a significant amount of time was spent on training the models, since the dependence on GAMS didn't allow training the reinforcement agent on a compute cluster with hundreds of cores, but we were rather constrained to a personal computer with 6 cores. This was a significant limitation, since training reinforcement agents is usually very computationally intensive (e.g. the state of the art models mentioned in the beginning of Chapter \ref{chapter3} were usually trained for months on hundreds of machines). We must note that training each agent took over a day of runtime on a personal computer.


\chapter*{Conclusion}
\addcontentsline{toc}{chapter}{Conclusion}
In this thesis, we explored the dependence of multistage scenario models on the chosen scenario tree structure.

We explored the dependence of the objective value on the scenario tree structure and we have shown that the structure of scenario trees that is usually used in practice may not always be the best. We found out that there are significant differences in the obtained rewards when using a different number of stages, while the differences between different scenario trees with a fixed number of stages are not that staggering.

Further we proposed an experiment to explore the dependence of the value of the objective function of the mean-CVaR model on the structure of a scenario tree that was built using the moment matching method from historical data using reinforcement learning. Further, we trained several reinforcement agents and evaluated their performance, finding that it is possible to train such an agent to aid in the task of choosing a scenario tree structure and we have shown that for 3 and 5 stage trees, the structure of the scenario trees chosen by the agent is rather different compared to the structure usually used in practice. We further explored the effect of including a penalty for choosing a scenario tree structure that is too complex. 

This thesis could be extended in several ways. Due to computational limitations, we had to constrain ourselves only to a small set of trees. It would be interesting to extend the experiment such that more stages and more branchings are allowed. Moreover, other scenario tree building methods than moment matching could be explored. It would be also very interesting to train the agent for a lot longer than we did, as we cannot be sure that a significantly longer training would not lead to better results. All of these extensions would however come at a significant computational cost with the current implementation.

%%% Bibliography
\include{bibliography}

%%% Figures used in the thesis (consider if this is needed)
%\listoffigures

%%% Tables used in the thesis (consider if this is needed)
%%% In mathematical theses, it could be better to move the list of tables to the beginning of the thesis.
%\listoftables

%%% Abbreviations used in the thesis, if any, including their explanation
%%% In mathematical theses, it could be better to move the list of abbreviations to the beginning of the thesis.
%\chapwithtoc{List of Abbreviations}

%%% Attachments to the master thesis, if any. Each attachment must be
%%% referred to at least once from the text of the thesis. Attachments
%%% are numbered.
%%%
%%% The printed version should preferably contain attachments, which can be
%%% read (additional tables and charts, supplementary text, examples of
%%% program output, etc.). The electronic version is more suited for attachments
%%% which will likely be used in an electronic form rather than read (program
%%% source code, data files, interactive charts, etc.). Electronic attachments
%%% should be uploaded to SIS and optionally also included in the thesis on a~CD/DVD.
%%% Allowed file formats are specified in provision of the rector no. 72/2017.

\appendix
\chapter*{Appendix}
\stepcounter{chapter}
\addcontentsline{toc}{chapter}{Appendix}
\section{Definitions of asset sets}
\begin{center}
\begin{table}[H]
\centering
\begin{tabular}{lllllll}
\toprule
 ACGL &   AFL &  AIG &   AJG &   ALL &   AON &   AXP \\
  BAC &   BEN &   BK &   BLK &   BRO &     C &    CB \\
 CINF &   CMA &  COF &   FDS &  FITB &    GL &    GS \\
 HBAN &   HIG &  IVZ &   JPM &   KEY &     L &   LNC \\
   MCO &   MMC &   MS &   MTB &  NTRS &   PGR &   PNC \\
   RE &    RF &  RJF &  SCHW &  SIVB &  SPGI &   STT \\
   TFC &  TROW &  TRV &   USB &   WFC &   WRB &  ZION \\
   \bottomrule
\end{tabular}
\caption{Stock tickers used in Chapter \ref{chapter4}.}
\label{table:stock_tickers_used}
\end{table}

\begin{table}[H]
\centering
\begin{tabular}{llllll}
\toprule
ALL &   BK &  ACGL &  FDS &  ZION &  TROW \\
 STT &  RJF &  SPGI &  AON &   LNC &   USB \\
 AXP &  AIG &   AJG &  BLK &    RE &  SIVB \\
 NTRS &   CB &   WRB &  BRO &     L &   IVZ \\
  BAC &   GS &   WFC &  MTB &   MMC &   BEN \\
  \bottomrule
\end{tabular}
\caption{Stock tickers in set $A$.}
\label{table:stock_tickers_in_set_A}
\end{table}


\begin{table}[H]
\centering
\begin{tabular}{llll}
\toprule
 HBAN &   COF &  JPM &  AFL \\
 PGR &   CMA &   RF &  TFC \\
 CINF &   MCO &   MS &    C \\
  HIG &  FITB &  TRV &   GL \\
 SCHW &   PNC &  KEY &     \\
 \bottomrule
\end{tabular}
\caption{Stock tickers in set $B$.}
\label{table:stock_tickers_in_set_B}
\end{table}

\begin{table}[H]
\centering
\scalebox{0.75}{
\begin{tabular}{l|llllllllll}
\toprule
Asset set 1 & ALL & BK  & ACGL & FDS & ZION & TROW & STT & RJF  & SPGI & AON \\
\toprule
Asset set 2 & LNC & USB & AXP  & AIG & AJG  & BLK  & RE  & SIVB & NTRS & CB  \\
\toprule
Asset set 3 & WRB & BRO & L    & IVZ & BAC  & GS   & WFC & MTB  & MMC  & BEN \\
\bottomrule
\end{tabular}
}
\caption{Specification of which stock tickers belong to which asset set.}
\label{table:asset_sets}
\end{table}
\end{center}

\section{Electronic attachment}
The Python source code used to obtain the results in Chapter \ref{chapter4} is included in the electronic attachment.
\openright
\end{document}
